% !TEX root =  master.tex
\chapter{Docker als Softwarecontainer (Moritz Werr)}

Für die Containerisierung wurde Docker verwendet. Softwarecontainer bieten viele Vorteile gegenüber traditioneller Virtualisierung. Traditionelle Virtualisierung wird verwendet um Ressourcen effizienter zu nutzen. Dabei werden viele schwache Maschinen in einer großen Maschine konsolidiert \cite[Vgl. S.152443]{watada_emerging_2019}. Diese teilen sich alle die gleichen Hardwareressourcen, welche über den Hypervisor auf die einzelnen virtuellen Maschinen verteilt wird. Allerdings teilen sich virtuelle Maschinen nicht ihre Softwarekomponenten. Dadurch wird oftmals der gleiche Programmcode öfters im Arbeitsspeicher vorkommen. Softwarecontainer funktionieren in dieser Hinsicht anders. Sie teilen sich normalerweise den Kernel mit dem Containerhost \cite[Vgl. S.152444]{watada_emerging_2019}. Diese sind dadurch um einiges leichtgewichtiger und schneller als traditionelle Virtualisierung \cite[Vgl. S.137]{fava_assessing_2024}. 
Aufgrund dieser Leichtgewichtigkeit und Geschwindigkeit lassen sich Docker-Anwendungen mit Orchestrierungswerkzeugen wie Kubernetes auch sehr leicht horizontal und vertikal skalieren \cite[Vgl. S.152458]{watada_emerging_2019}. 

Weiterhin sind Anwendungen innerhalb von Docker-Containern auch sehr portabel, da ein Container nicht vom Hostsystem abhängig ist, sondern von dessen Docker-Image. Alle Abhängigkeiten (Dependencies) sind in einem einzelnen Image gebündelt  \cite[Vgl. S.137]{fava_assessing_2024}. Dadurch sind containerisierte Anwendungen hoch portabel. Sie können auf jeder Maschine laufen, die eine kompatible Docker-Engine hat. Nicht nur läuft die Anwendung auf sehr vielen Maschinen, sondern es läuft auch überall die gleiche Version. Das Image sorgt für einen hohen Grad an Konsistenz und Wiederholbarkeit.

Außerhalb der hohen Effizienz und Portabilität von Softwarecontainern, bieten diese auch noch einen gewissen Grad an Sicherheit. Obwohl virtuelle Maschinen einen höheren Grad an Sicherheit bieten durch getrennte Kernel \cite[Vgl. S.152455]{watada_emerging_2019}, bieten Container einen guten Schutz durch Prozessisolierung, Dateisystemisolierung, Ressourcenlimitierung und Netzwerkisolation \cite[Vgl. S.152444ff]{watada_emerging_2019}. Dafür werden Funktionalitäten des Linux Kernels verwendet. Primär wären dies CGroups, Namespaces und MAC-Systemen (Mandatory Access Control) wie SELinux oder Apparmor. Cgroup sorgen dabei für die Resourcenlimitierung, Namespaces für die Prozess- und Netzwerkisolation und Apparmor für die Dateisystemisolierung.

Durch die Nutzung von Docker ist die Anwendung sehr performant, portabel und auch noch gut abgesichert. 