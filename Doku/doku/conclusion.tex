% !TEX root =  master.tex
\chapter{Zusammenfassung (Max Stege)}

\section{Fazit (Max Stege)}

Das vorliegende Projekt hat die Entwicklung der Intel Core i7-Prozessorreihe analysiert, um zentrale Fragestellungen der modernen Prozessorentwicklung zu beantworten. Dabei wurden Daten von der Herstellerwebsite erhoben, bereinigt, in einer relationalen Datenbank gespeichert und schließlich visualisiert. Im Fokus standen die Überprüfung des Mooreschen Gesetzes, die Identifikation von Metriken mit signifikanten Leistungssteigerungen und die Analyse von Bereichen, die eine Stagnation aufweisen.

Die Ergebnisse zeigen, dass das Mooresche Gesetz zwar in seinem ursprünglichen Sinne an Relevanz verloren hat, jedoch weiterhin als Leitfaden für technologische Entwicklungen dient. Die Leistungssteigerung moderner Prozessoren wird weniger durch höhere Taktraten erreicht – die sich zunehmend den physikalischen Grenzen nähern – als vielmehr durch eine höhere Kernanzahl, fortschrittliche Fertigungstechnologien und spezialisierte Architekturansätze. Stagnation zeigte sich vor allem in Metriken wie den Taktraten und der klassischen Single-Thread-Leistung, was auf thermische und energetische Beschränkungen zurückzuführen ist.

Dieses Projekt war jedoch nicht frei von Herausforderungen. Der Aufwand für die Umstrukturierung der CSV-Dateien und die Filterung irrelevanter Informationen war beträchtlich. Zudem war die Datenqualität nicht immer zufriedenstellend, da einige Werte unvollständig oder uneinheitlich waren, was zusätzliche Bereinigungsmaßnahmen erforderlich machte.

Ein weiteres Problem ergab sich bei der Auswahl geeigneter Analysemethoden und Visualisierungen. Trotz umfangreicher Bereinigungen blieben in den Daten leichte Inkonsistenzen, die die Interpretierbarkeit bestimmter Trends erschwerten. Diese Schwierigkeiten verdeutlichen, wie entscheidend eine qualitativ hochwertige und standardisierte Datenbasis für Datenanalysen ist.

Insgesamt konnte das Projekt jedoch eine robuste Pipeline zur Datenaufbereitung und -analyse etablieren, die als Grundlage für ähnliche Untersuchungen dienen kann. Die gewonnenen Erkenntnisse geben Einblicke in die technologische Entwicklung der Prozessoren und legen die Grundlage für weiterführende Fragestellungen.

\section{Ausblick (Max Stege)}

Die bei diesem Projekt aufgetretenen Schwierigkeiten bieten zugleich Ansatzpunkte für zukünftige Arbeiten. Insbesondere die Automatisierung der Datenbeschaffung, etwa durch Web-Scraping-Techniken oder das Schaffen standardisierter APIs, könnte den Arbeitsaufwand reduzieren und die Datenqualität verbessern. Ein weiterer wichtiger Schritt wäre die Integration von Validierungsmechanismen, um die Konsistenz der Daten bereits während der Bereinigungsphase sicherzustellen.

Zukunftsgerichtete Studien könnten sich mit der Analyse neuerer Technologien befassen, wie etwa der 3D-Chip-Architektur, Fortschritten in der Materialforschung (z.\,B. Graphen) oder dem wachsenden Einfluss spezialisierter Rechenmodule für Künstliche Intelligenz und maschinelles Lernen. Dabei könnten auch Predictive-Analytics-Methoden eingesetzt werden, um die Trends der kommenden Jahre zu prognostizieren.

Ein weiterer interessanter Ansatz wäre der Vergleich der Entwicklungen bei anderen Prozessorherstellern wie AMD oder Apple, um unterschiedliche Innovationsstrategien und deren Auswirkungen auf den Markt zu bewerten. Zudem könnten breiter angelegte Analysen mit zusätzlichen Metriken durchgeführt werden, um ein umfassenderes Bild der technischen Fortschritte in der Halbleiterbranche zu erhalten.

Schließlich wäre eine engere Verknüpfung der Datenanalyse mit ökologischen Aspekten von Interesse. Der Energieverbrauch und die Nachhaltigkeit der Prozessorfertigung könnten entscheidende Faktoren für die zukünftige Entwicklung der Branche werden.

Zusammenfassend lässt sich sagen, dass dieses Projekt nicht nur wertvolle Einblicke in die Entwicklung moderner Prozessoren geliefert hat, sondern auch weitere Impulse für zukünftige Untersuchungen gibt. 


