% !TEX root =  master.tex
\chapter{Die Webanwendung (Phil Richter)}
Die Webanwendung besteht aus einem Internetforum, in dem Benutzer Beiträge erstellen, kommentieren und liken
können. Um am Forum teilzunehmen, muss sich der Benutzer mit einer Emailadresse und einem Passwort registrieren und einloggen.
Falls der Benutzer sein Passwort vergisst, kann er dieses mit Hilfer einer Emailadresse zurücksetzen und ein neues vergeben. 
Des Weiteren existiert ein Admin Account mit erweiterten Rechten. Dieser kann Beiträge von allen Nutzern
löschen, während normale Benutzer nur ihre eigenen Beiträge löschen können. Zudem kann dieser sowohl die Emailadresse als auch
das Passwort eines Benutzers ändern. Falls notwendig kann ein Admin auch einen Account löschen.

Die Anwendung besteht im Frontend aus HTML \cite{noauthor_html_nodate}, CSS \cite{noauthor_cascading_nodate} und JavaScript
\cite{noauthor_learn_nodate}. Das Backend wurde mit PHP \cite{noauthor_php_2024} geschrieben. Zusätzlich existiert noch eine
Datenbank, in der die Benutzerdaten und Beiträge gespeichert werden. Die Datenbank wurde mit MariaDB \cite{noauthor_mariadb_nodate}
realisiert. Die Mails werden mit msmtp \cite{noauthor_msmtp_nodate} über einen SMTP-Server von Gmail versendet.

\chapter{Webserver}
Für den Webserver wurde ein Dockerfile implementiert welches wiefolgt aussieht:
\begin{lstlisting}[caption={\texttt{dockerfile}},captionpos=b, breaklines=true]
    FROM php:8.1-apache
    COPY src/ /var/www/html
    RUN apt-get update && apt-get install -y msmtp && rm -rf /var/cache/apt/lists
    COPY msmtprc /etc/msmtprc
    RUN chmod 600 /etc/msmtprc && chown www-data: /etc/msmtprc
    RUN mkdir /var/www/html/pictures
    RUN sed -i 's/AllowOverride None/AllowOverride All/g' /etc/apache2/apache2.conf
    RUN printf "AddType application/x-httpd-php .html\nAddType application/x-httpd-php .htm\n" >> /etc/apache2/apache2.conf
    RUN a2enmod rewrite
    RUN chown www-data: -R /var/www/html/
    RUN docker-php-ext-install mysqli && docker-php-ext-enable mysqli
    RUN echo "sendmail_path = /usr/bin/msmtp -t" >> /usr/local/etc/php/php.ini-development
    RUN cp /usr/local/etc/php/php.ini-development /usr/local/etc/php/php.ini
    CMD apachectl -DFOREGROUND
\end{lstlisting}

Im ersten Schritt wird in Zeile 1 das Basisimage \lstinline|php: 8.1-apache| gesetzt. Dies ist ein bereis vorkonfiguriertes
Image mit PHP 8.1 und Apache Webserver. In Zeile 2 wird der Quellcode der Webanwendung, welche im Ordner \lstinline|src/| liegt,
in den Ordner \lstinline|/var/www/html| des Containers kopiert. Damit die Webanwendung erfolgreich ausgefürt wird, müssen
verschiedene Konfigurationen von Apache vorgenommen werden. Dies passiert in Zeile 7 bis 9. Dabei wird unteranderem das Apache
Modul \lstinline|rewrite| aktiviert, sowie angepasst, dass die Dateiendungen \lstinline|.html| und \lstinline|.htm| als PHP-Dateien
interpretiert werden.


Zusätzlich werden weitere, für die Webanwendung notwendige Komponenten installiert. Dazu gehört für den Mailversand
\lstinline|msmtp| in Zeile 3 und für die Datenbank \lstinline|mysqli| in Zeile 11. Damit msmtp verwendet wird,
wird in Zeile 12 der \lstinline|sendmail_path| in der PHP-Konfiguration korrekt gesetzt. Zudem wird in Zeile 4
die Konfigurationsdatei \lstinline|msmtprc| in den Container kopiert. Diese Datei enthält die Konfigurationen wie unter
anderem Host und Port für den Mailversand.

Des Weitren wird in Zeile 6 ein Ordner \lstinline|pictures| erstellt, in welchem später die von den Nutzern hochgelandenen
Bilder gespeichert werden. In Zeile 5 und 10 wird der Besitzer der Dateien  und Ordner auf \lstinline|www-data| gesetzt.
Dadurch wird sichergestellt, dass der Apache Webserver auf die Dateien zugreifen kann.
Zum Schluss wird in Zeile 14 der Apache Webserver gestartet.


Um den Webserver als eigenständigen Container zu betreiben wird folgender Code in der \lstinline|docker-compose.yml| Datei
definiert:
\begin{lstlisting}[caption={\texttt{docker-compose.yml}},captionpos=b]
    webserver:
    image: webserver
    build: .
    volumes:
      - pictures:/var/www/html/pictures
    labels:
      - "traefik.enable=true"
      - "traefik.http.routers.webserver.rule=Host(`localhost`)"
      - "traefik.http.services.webserver.loadbalancer.server.port=80"
      - "traefik.http.routers.webserver.entrypoints=web,websecure"
      - "traefik.http.routers.webserver.tls=true"
    environment:
      DATABASE_USER: web_eng_user
      DATABASE_HOST: mariadb
      DATABASE_PASS: linux
      DATABASE_NAME: web_eng
      GMAIL_APP_PASSWORD: ${GMAIL_APP_PASSWORD}
    depends_on:
      db:
        condition: service_healthy
        restart: true
    healthcheck:
      test: ["CMD", "curl", "-f", "http://localhost"]
      interval: 10s
      timeout: 5s
      retries: 3
      start_period: 10s
\end{lstlisting}

Dabei wird in Zeile 2 das Image \lstinline|webserver| definiert, welches in Zeile 3 aus dem aktuellen Verzeichnis, dem dockerfile,
gebaut wird. In Zeile 5 wird ein Volume \lstinline|pictures| definiert. Dies ist dazu notwendig, dass die hochgeladenen Bilder
auch nach einem Neustart des Containers noch vorhanden sind. Die in Zeile 6 und folgenden definierten Labels sind dazu notwendig,
dass der Webserver über den TLS-Proxy \lstinline|traefik| erreichbar ist. Außerdem wird festgelegt, dass der Webserver auf
Localhost erreichbar ist und auf Port 80 lauscht. Zusätzlich wird tls aktiviert.

Um auf die Datenbank zugreifen zu können und Mails über Gmail versenden zu können, werden in Zeile 12 und folgende 
die notwendigen Umgebungsvariablen definiert und gesetzt. Dabei wird das Passwort für den Gmail Account aus einer
Lokalen .env Datei gelesen. Somit wird sichergestellt, dass das Passwort nicht im Klartext vorliegt.

Da der Webserver auf die Datenbank angewiesen ist, wird in Zeile 18 bis 21 eine Abhängigkeit definiert. Dabei
wird überprüft, ob die Datenbank erreichbar ist. Falls nicht, wird diese neu gestartet. Des Weiteren wird in Zeile 22 bis 27
ein Healthcheck definiert. Dabei wird alle 10 Sekunden mit Hilfe eines Curl-Befehls überprüft, ob der Webserver erreichbar ist.
Falls der Healthcheck dreimal fehlschlägt, gilt der Container als "unhealthy".