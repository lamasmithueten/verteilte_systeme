% !TEX root =  master.tex
\chapter{Docker Compose (Max Stege)}

Docker Compose ist ein essentielles Werkzeug zur Verwaltung von Multi-Container-Anwendungen. Es bietet eine Vielzahl von Vorteilen, die den Entwicklungs- und Betriebssprozess erheblich vereinfachen:

\begin{enumerate}
\item \textbf{Multi-Container-Verwaltung}: \
Docker Compose erlaubt es, mehrere Container mit einer einzigen YAML-Datei zu orchestrieren. Dies ist besonders nützlich in komplexen Anwendungen wie der hier entwickelten Drei-Schichten-Anwendung, wo Backend, Frontend und Datenbank in getrennten Containern laufen \cite{docker_compose_docs}.

\item \textbf{Einfachere Konfiguration}: \
Anstatt Container manuell mit zahlreichen \texttt{docker run}-Befehlen zu starten, wird die gesamte Konfiguration in einer zentralen Datei (z. B. \texttt{docker-compose.yml}) definiert. Dies umfasst Details wie Netzwerkverbindungen, Umgebungsvariablen, Abhängigkeiten und Healthchecks \cite{fava2024}.

\item \textbf{Übersichtlichkeit}: \
Docker Compose bietet eine klare Struktur für komplexe Anwendungen. In der YAML-Datei können Dienste wie der Webserver, die Datenbank und der TLS-Proxy logisch voneinander getrennt definiert werden, was die Wartbarkeit erleichtert \cite{watada2019}.

\item \textbf{Wartbarkeit}: \
Durch die zentrale Definition aller Dienste wird der Wartungsaufwand minimiert. Änderungen, wie das Hinzufügen neuer Dienste oder das Anpassen von Konfigurationen, können unkompliziert in der YAML-Datei vorgenommen werden \cite{fava2024}.

\item \textbf{Bessere Bedienbarkeit}: \
Mit einfachen Befehlen wie \texttt{docker-compose up} oder \texttt{docker-compose down} können alle Container gleichzeitig gestartet oder gestoppt werden. Zusätzlich ermöglicht die Option \texttt{--build}, neue Container-Builds direkt aus dem Quellcode zu erstellen \cite{docker_compose_docs}.
\end{enumerate}

In der entwickelten Webanwendung wurde Docker Compose eingesetzt, um die Komponenten Webserver, Datenbank und TLS-Proxy zu koordinieren:

\begin{itemize}
\item \textbf{Webserver}: \
Das Docker Compose-Setup definiert das Image, die Volumes für Bilder, Labels für die TLS-Integration über den Traefik-Proxy und wichtige Umgebungsvariablen, die die Verbindung zur Datenbank regeln. Healthchecks stellen sicher, dass der Webserver ordnungsgemäß funktioniert \cite{services_docs}.

\item \textbf{Datenbank}: \
Die MariaDB-Datenbank wird ebenfalls als eigenständiger Dienst mit spezifischen Umgebungsvariablen und Volumes betrieben. Ein Initialisierungsskript wird beim Start des Containers ausgeführt, um die Datenbankstruktur zu erstellen \cite{watada2019}.

\item \textbf{TLS-Proxy}: \
Der Traefik-Container dient als Proxy, der sowohl HTTP- als auch HTTPS-Anfragen handhabt. Hierfür werden spezifische Ports und Konfigurationsdateien genutzt, um eine sichere Datenübertragung zu gewährleisten \cite{docker_compose_docs}.
\end{itemize}

Die Verwendung von Docker Compose bietet nicht nur Effizienz und Konsistenz bei der Entwicklung, sondern verbessert auch die Skalierbarkeit und Stabilität der Anwendung im Produktivbetrieb.