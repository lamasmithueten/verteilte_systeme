\chapter{Datenbank (Max Stege)}

Die Datenbank ist eine der zentralen Komponenten der Drei-Schichten-Architektur. In diesem Projekt wurde MariaDB als Datenbankserver eingesetzt, der mithilfe von Docker Compose als eigenständiger Container betrieben wird. Die Konfiguration des Datenbankdienstes in der Docker-Compose-Datei sieht wie folgt aus:

\lstset{
breaklines=true,         % Enable line wrapping
breakatwhitespace=false, % Allow breaks at any character (not just whitespace)
basicstyle=\ttfamily,    % Use monospaced font
}

\begin{lstlisting}[caption={\texttt{docker-compose.yml}},captionpos=b]
db:
    image: mariadb
    hostname: mariadb
    restart: always
    volumes:
      - db:/var/lib/mysql
      - ./database.sql:/docker-entrypoint-initdb.d/database.sql
    environment:
      MARIADB_USER: web_eng_user
      MARIADB_PASSWORD: linux
      MYSQL_ROOT_PASSWORD: example
    healthcheck:
      test: ["CMD", "healthcheck.sh", "--connect", "--innodb_initialized"]
      start_period: 10s
      interval: 10s
      timeout: 5s
      retries: 3
\end{lstlisting}

Der Datenbankcontainer verwendet das MariaDB-Image, ein populäres und performantes Open-Source-Datenbankmanagementsystem \cite{mariadb_docs}. Der Hostname des Containers wird explizit auf \texttt{mariadb} gesetzt, um eine klare Identifikation im Docker-Netzwerk zu ermöglichen. Das Volumemapping \texttt{db:/var/lib/mysql} sorgt dafür, dass die Datenbank persistent bleibt und bei einem Neustart des Containers erhalten bleibt \cite{docker_persistence}. Zusätzlich wird ein SQL-Skript \texttt{database.sql} eingebunden, das die initiale Struktur der Datenbank definiert.

Um die Datenbank zu konfigurieren, werden verschiedene Umgebungsvariablen definiert:
\begin{itemize}
\item \texttt{MARIADB\_USER}: Der Benutzername für den Datenbankzugriff.
\item \texttt{MARIADB\_PASSWORD}: Das Passwort für den definierten Benutzer.
\item \texttt{MYSQL\_ROOT\_PASSWORD}: Das Passwort für den Root-Benutzer der Datenbank.
\end{itemize}

Diese Variablen ermöglichen eine flexible Anpassung der Datenbankkonfiguration, ohne Änderungen am Containerimage vorzunehmen \cite{environment_variables}.

Ein Healthcheck wird definiert, um sicherzustellen, dass die Datenbank ordnungsgemäß funktioniert. Der Test prüft, ob die Datenbank erreichbar ist und ob das InnoDB-Subsystem korrekt initialisiert wurde \cite{docker_healthcheck}. Die Parameter \texttt{start\_period}, \texttt{interval}, \texttt{timeout} und \texttt{retries} legen fest, wie oft und in welchen Zeitabständen der Healthcheck durchgeführt wird.

Die Datenbank ist essentiell für die persistente Speicherung der Anwendungsdaten. Der Webserver kommuniziert direkt mit der Datenbank, um Benutzerinformationen, Forenbeiträge und andere relevante Daten zu speichern und abzurufen \cite{mariadb_communication}. Dank der Containerisierung kann die Datenbank isoliert und unabhängig von anderen Komponenten betrieben werden, was die Wartung und Skalierung erleichtert \cite{watada2019}.