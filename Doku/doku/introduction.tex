% !TEX root =  master.tex
\chapter{Einleitung (Moritz Werr)}

\nocite{*}

Im Rahmen dieses Projektes wurde eine Anwendung in drei Schichten aufgeteilt mithilfe von Docker und Docker-Compose. Diese wurde im Rahmen eines vorherigen Semesters implementiert als eine monolithische Serveranwendung, obwohl diese jedoch eigentlich aus mehreren Komponenten besteht. Generell besteht diese aus drei Teilen: dem Webserver und der Programmierlogik für die Anwendung, einem Datenbankserver für die persistente Speicherung der Daten und einer TLS-Verschlüsselung für die Sicherheit der Datenübertragung. Vorher liefen diese alle auf einem Server, es war eine monolithische Architektur. Nun wird diese in ihren einzelnen Funktionalitäten aufgespalten, hin zu einer Mikroservicearchitektur. Jede einzelne Funktionalität wird in einen eigenen Softwarecontainer übertragen und laufen isoliert auf der gleichen Maschine, indem sie sich den Betriebsystemkernel teilen \cite[Vgl. S. 152444]{watada_emerging_2019}. Dieser Prozess nennt sich Containerisierung.

Durch die Umwandlung in eine Mikroservicearchitektur wird ein hoher Grad an Flexibilität gewonnen. Die einzelnen Komponenten arbeiten unabhängig voneinander und ergeben gemeinsam ein funktionierendes System \cite[Vgl. S.1f]{fowler_microservices_2015}. Dafür muss die Anwendung jedoch in manchen Fällen  umgeschrieben werden. Durch den Umbau zu einer Mikroservicearchitektur, wird oftmals die Komplexität der Anwendung erhöht \cite[Vgl. S.10]{su_modular_2024}, weswegen noch weitere Werkzeuge zur Verwaltung der Anwendung benötigt werden. Hilfswerkzeuge zur Verwaltung und Konfiguration der Anwendung werden benötigt. 

Weiterhin sind nicht nur die einzelnen Komponenten austauschbar, sondern können nun auch weitere Funktionalitäten durch die Containerisierung erreicht werden. Es können sehr einfach Healthchecks implementiert werden, die üverprüfen, ob einzelne Komponenten noch richtig funktionieren \cite{noauthor_services_0100}. Wenn dies nicht mehr der Fall ist, können die einzelnen Komponenten der drei Schichten neugestartet werden \cite{noauthor_services_0100}. Dafür muss definiert werden wie diese miteinander zusammenarbeiten. Wenn nun ein einzelner Container Probleme hat, wird dies erkannt und er kann automatisch neugestartet werden. Weiterhin sorgt dies auch dafür, dass Container in der richtigen Reihenfolge gestartet werden \cite{noauthor_services_0100}.

Im Rahmen dieses Projektes wurden Docker und Docker-Compose verwendet aufgrund ihrer guten Dokumentation und großen Anzahl an Features. Kernbestandteil war Docker zur Containerisierung und Docker-Compose zur Verwaltung der Container. Es musste ein neues Dockerfile für den Webserver erstellt werden, zwei vorhandene Containerimages wurden für die TLS-Verschlüsselung bzw. die Datenspeicherung verwendet und ein Initialisierungsskript für die Datenbank wurde erstellt.